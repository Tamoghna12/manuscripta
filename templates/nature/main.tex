\documentclass[11pt]{article}
\usepackage[margin=1in]{geometry}
\usepackage{amsmath,amssymb}
\usepackage{graphicx}
\usepackage{hyperref}
\usepackage{times}
\usepackage{booktabs}
\usepackage{setspace}
\usepackage{lineno}
\usepackage[super,sort&compress]{natbib}

\linenumbers
\doublespacing

\title{\textbf{Paper Title for Nature}}
\author{
  Author One\textsuperscript{1*},
  Author Two\textsuperscript{2},
  Author Three\textsuperscript{1,2} \\[6pt]
  \small\textsuperscript{1}Department, University One, City, Country \\
  \small\textsuperscript{2}Department, University Two, City, Country \\
  \small\textsuperscript{*}Corresponding author. Email: author1@example.com
}
\date{}

\begin{document}
\maketitle

\begin{abstract}
\noindent
Your abstract here (max 150 words for Nature). Nature articles have structured prose abstracts without headings.
\end{abstract}

\section*{Main}

Introduction text goes here. Nature articles do not use numbered section headings.

\textbf{Results.}

\textbf{Sub-result one.}

\textbf{Sub-result two.}

\textbf{Discussion.}

\section*{Methods}
\subsection*{Data collection}
\subsection*{Statistical analysis}
\subsection*{Code availability}
\subsection*{Data availability}

\section*{Acknowledgements}
\section*{Author contributions}
\section*{Competing interests}
The authors declare no competing interests.

\bibliographystyle{naturemag}
\bibliography{references}

\section*{Extended Data}
\section*{Supplementary Information}

\end{document}
