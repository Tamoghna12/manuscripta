\documentclass[11pt,a4paper]{article}
\usepackage[margin=1in]{geometry}
\usepackage{amsmath,amssymb,amsthm}
\usepackage{graphicx}
\usepackage{hyperref}
\usepackage{enumitem}
\usepackage{fancyhdr}
\usepackage{listings}
\usepackage{xcolor}

\pagestyle{fancy}
\fancyhf{}
\lhead{Course Name -- Homework X}
\rhead{Your Name}
\cfoot{\thepage}

\newcommand{\problem}[1]{\section*{Problem #1}}
\newcommand{\solution}{\noindent\textbf{Solution.}\quad}

\lstset{basicstyle=\ttfamily\small,breaklines=true,frame=single,backgroundcolor=\color{gray!10}}

\title{
  \textbf{Course Code: Course Name} \\
  Homework Assignment X
}
\author{
  Your Name \\
  Student ID: 12345678 \\
  Collaborators: None
}
\date{Due: Month Day, Year}

\begin{document}
\maketitle

\problem{1}
\textit{Problem statement goes here.}

\solution
Write your solution here.

\begin{align}
  f(x) &= x^2 + 2x + 1 \\
  &= (x + 1)^2
\end{align}

\problem{2}
\textit{Problem statement goes here.}

\solution

\begin{proof}
We proceed by induction.

\textbf{Base case:} When $n = 1$, the claim holds.

\textbf{Inductive step:} Assume the claim holds for $n = k$. We show it holds for $n = k + 1$.

Therefore, by induction, the claim holds for all $n \geq 1$.
\end{proof}

\problem{3}
\textit{Problem statement goes here.}

\solution

\begin{lstlisting}[language=Python]
def solve(n):
    if n <= 1:
        return n
    return solve(n-1) + solve(n-2)
\end{lstlisting}

\problem{4}
\textit{Problem statement goes here.}

\solution
Your solution here.

\end{document}
