\documentclass[runningheads]{llncs}
\usepackage{amsmath,amssymb}
\usepackage{graphicx}
\usepackage{hyperref}
\usepackage{booktabs}
\usepackage{cite}

\begin{document}

\title{Paper Title for Springer LNCS}
\titlerunning{Short Title}

\author{Author One\inst{1} \and Author Two\inst{2}}

\authorrunning{A. One et al.}

\institute{University One, City, Country \\
\email{author1@example.com} \and
University Two, City, Country \\
\email{author2@example.com}}

\maketitle

\begin{abstract}
Your abstract here. Springer LNCS (Lecture Notes in Computer Science) is used for many CS conferences.
\keywords{Keyword one \and Keyword two \and Keyword three}
\end{abstract}

\section{Introduction}
\section{Related Work}
\section{Approach}
\subsection{Overview}
\subsection{Details}
\section{Evaluation}
\subsection{Experimental Setup}
\subsection{Results}
\subsection{Discussion}
\section{Conclusion}

\begin{acknowledgement}
This work was supported by\ldots
\end{acknowledgement}

\bibliographystyle{splncs04}
\bibliography{references}

\end{document}
